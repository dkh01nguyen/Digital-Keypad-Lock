\section{Thiết kế}
\begin{figure}[H]
    \centering
    \includegraphics[scale = 1.2]{picture/blockd.png}
    \caption{Sơ đồ khối tổng quát}
    \label{block-diagram}
\end{figure}

Sơ đồ bao gồm bốn khối: \textit{Khối xử lý trung tâm, khối giao diện người dùng, khối truyền động và cảm biến và quản lý năng lượng}.
\begin{enumerate}[label = \textbf{\arabic*}.]
    \item \textbf{Khối xử lý trung tâm:}
        \vspace{-0.5cm}
        \begin{itemize}[label = -]
            \item Đây là đầu não hành vi của toàn bộ ổ khóa. Vi điều khiển thực hiện nhiệm vụ tiếp nhận tín hiệu đầu vào, 
                xử lý thuật toán xác thực và phát lệnh điều khiển đến các thiết bị ngoại vi.
            \item Khi nhập mật khẩu, MCU nhận chuỗi ký tự từ bàn phím, sau đó thực hiện tìm kiếm mật khẩu thực liên tục trong chuỗi 
                dữ liệu nhập vào bằng thuật toán xác thực và kích hoạt cơ cấu mở khóa nếu nội dung được kiểm tra là chính xác.
            \item Trong trường hợp người dùng sử dụng phương thức mở khóa bằng chìa cơ, MCU sẽ đọc trạng thái từ 
                microswitch gắn với ổ khóa cơ học để thực hiện hành động tương ứng để mở khóa.
            \item Sau khi nhập mật khẩu, MCU sẽ phát lệnh điều khiển đến các thiết bị đầu ra như màn hình hiển thị, còi báo động
                và relay kích hoạt solenoid ứng với từng trạng thái hoạt động của hệ thống như: tiến hành mở khóa khi đã nhập đúng;
                sai mật khẩu nhiều lần; hiển thị trực quan và bảo mật nội dung đang nhập...
            \item Song song với việc xử lý mở khóa, MCU đảm nhận cả việc giám sát và quản lý các thông số vận hành quan trọng bao gồm: hiển thị trực 
                quan trạng thái khóa; hiển thị thông báo người dùng khi điện áp nguồn không ổn định...
        \end{itemize}
    \item \textbf{Khối giao diện người dùng:}
         \begin{itemize}[label = -]
            \item \textbf{Về thiết bị đầu vào:} bàn phím ma trận kích thước 4 $\times$ 3 cho phép người dùng 
                nhập mật khẩu và các lệnh điều khiển thông qua các phím số từ "0" đến "9", phím “Enter” để xác nhận, phím “Delete” 
                để xóa ký tự gần nhất, đồng thời các ký tự chữ cái từ “A” đến “F” được tích hợp trên các phím số từ “1” đến “6”.
            \item Một microswitch cơ học được lắp đặt tại vị trí ổ khóa cơ có chức năng phát hiện và thông báo cho MCU khi 
                người dùng thực hiện thao tác mở khóa bằng chìa khóa truyền thống. 
            \item Nút bấm có ký hiệu "Open" được gắn ở mặt trong của cửa, khi được kích hoạt sẽ truyền tín hiệu mở khóa đến MCU. 
            \item Về thiết bị đầu ra: màn hình hiển thị dạng LCD 16$\times$2 hoặc màn hình 
                TFT đảm nhiệm việc trình bày các giao diện người dùng và các thông báo hệ thống như chuỗi đang nhập, 
                kết quả xác thực, thiết lập mật khẩu mới, cảnh báo mức pin thấp,...
            \item Còi báo động phát ra các tín hiệu âm thanh trong các tình huống như phát hiện nhập sai mật khẩu liên tiếp nhiều 
                lần hoặc nhắc nhở người dùng đóng cửa sau khi ra vào.
            \item Hệ thống đèn LED chỉ thị trạng thái được đặt ở cả hai mặt trước và sau của khóa, giúp người dùng 
                quan sát trực quan trạng thái khóa của cửa.
        \end{itemize}
    \item \textbf{Khối truyền động và cảm biến:}
        \begin{itemize}[label = -]
            \item Khóa điện từ kiểu solenoid đóng vai trò là cơ cấu chấp hành chính, thực hiện động tác đóng và mở chốt 
                khóa theo chỉ thị từ vi điều khiển.
            \item Do solenoid yêu cầu nguồn cấp có công suất lớn hơn khả năng cung cấp trực tiếp của MCU, 
                vi điều khiển sẽ điều khiển solenoid gián tiếp thông qua một relay trung gian, tín hiệu điều khiển mức 
                logic 5V từ chân GPIO của MCU kích hoạt cuộn dây relay, khi relay đóng tiếp điểm sẽ cho phép dòng điện 12V 
                hoặc điện áp định mức phù hợp chạy qua solenoid để kích hoạt cơ chế khóa.
            \item Về phần cảm biến, hệ thống sử dụng một microswitch cảm biến trạng thái cửa nhận biết vị trí đóng hoặc mở 
                của cánh cửa và gửi về MCU, phục vụ cho thao tác cảnh báo người dùng.
        \end{itemize}
    \item \textbf{Khối quản lý năng lượng:}
         \begin{itemize}[label = -]
            \item Nguồn điện chính được cấp từ bộ pin kiểu AA gồm bốn viên mắc nối tiếp tạo ra điện áp khoảng 6V, đáp ứng nhu cầu
            năng lượng cho tất cả các module trong thiết bị. Nguồn pin này được đưa vào mạch nguồn của bo mạch và qua các 
            mạch điều chỉnh điện áp để tạo ra các mức điện áp ổn định 5V hoặc 3.3V cung cấp cho vi điều khiển cùng các linh kiện 
            điện tử khác.

            \item Để đảm bảo hệ thống hoạt động ổn định, khóa cần được trang bị mạch giám sát điện áp nguồn hoạt động trên 
            nguyên tắc đo điện áp bộ nguồn và chuyển đổi thành tín hiệu số gửi về MCU. Từ đó, đưa ra các thông báo pin yếu đến 
            người dùng khi điện áp pin thấp hơn định mức.

        \end{itemize}
\end{enumerate}

