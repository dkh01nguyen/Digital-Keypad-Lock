\section{Cấu tạo và Nguyên lý hoạt động}
\subsection{Nguyên lý hoạt động}
Ổ khóa điện tử cho phép người dùng mở khóa cửa bằng cách sử dụng mật khẩu hoặc sử dụng chìa khóa cơ. Quy trình xử lý đơn giản của ổ khóa 
được mô tả trong sơ đồ sau:
\begin{figure}[H]
    \centering
    \includegraphics[width=0.8\textwidth]{picture/basic-flowchart.png}
    \caption{Sơ đồ hoạt động của ổ khóa điện tử}
    \label{fig:flowchart}
\end{figure}
Người dùng sẽ nhập một chuỗi hai mươi ký tự trong đó có chuỗi con bốn ký tự là mật khẩu của hệ số thập lục phân. Sau đó, vi xử lý trung tâm sẽ tìm kiếm mật khẩu
trong chuỗi vừa nhập bằng thuật toán \code{Knutt - Morris - Pratt} và so sánh với dữ liệu được lưu trong bộ nhớ của vi. Nếu đúng, vi xử lý trung tâm
sẽ gửi tín hiệu để mở chốt khóa và ngược lại. Để thao tác dễ dàng hơn, người sử dụng cũng sẽ được thấy và được cảnh báo các trạng thái của ổ khóa thông qua một màn hình LCD.
\subsection{Linh kiện phần cứng}
\subsubsection{Vi điều khiển STM32F103C8T6}
Vi điều khiển STM32F103C8T6 thuộc dòng STM32F1 của STMicroelectronics, là một lựa chọn phổ biến cho 
các dự án hệ thống nhúng đòi hỏi sự cân bằng giữa hiệu năng, bộ nhớ, và chi phí. Thiết bị được tích hợp lõi Arm\textsuperscript{\textregistered} Cortex\textsuperscript{\textregistered}-M3 32-bit RISC hiệu năng cao hoạt động ở tần số 72MHz, 
bộ nhớ nhúng tốc độ cao và có nhiều chân I/O ngoại vi mở rộng, được kết nối qua hai bus APB. Vi điều khiển đóng vai trò chính trong việc nhận, xử lý và truyền dữ liệu thông qua giao thức GPIO và I\textsuperscript{2}C.

\textbf{Đặc điểm kỹ thuật:}
\begin{itemize}
    \vspace{-10pt}
    \item Lõi xử lý trung tâm: Arm\textsuperscript{\textregistered} Cortex\textsuperscript{\textregistered}-M3
        \begin{itemize}
            \vspace{-10pt}
            \item Tần số tối đa 72 MHz, hiệu suất truy cập bộ nhớ đạt 1.25 DMIPS/MHz (Dhrystone 2.1) khi không có chu kỳ đang chờ.
        \end{itemize}
    \vspace{-10pt}
    \item Bộ nhớ:
        \begin{itemize}
            \vspace{-10pt}
            \item Flash: 64KB
            \item SRAM: 20KB
        \end{itemize}
    \vspace{-10pt}
    \item Giao tiếp ngoại vi
        \vspace{-10pt}
        \begin{itemize}
        \item Tối đa 2 giao diện I\textsuperscript{2}C (hỗ trợ SMBus/PMBus).
        \item Tối đa 3 bộ USART (hỗ trợ giao diện ISO 7816, LIN, hồng ngoại IrDA, điều khiển modem).
        \item Tối đa 2 giao diện SPI (tốc độ lên đến 18 Mbit/s).
        \item Giao diện CAN (chuẩn 2.0B Active).
        \item Giao diện USB 2.0 tốc độ tối đa.
    \end{itemize}
    \item 51 cổng giao tiếp I/O.
\end{itemize}
\begin{figure}[H]
    \centering
    \includegraphics[scale = 0.6]{picture/stm32.jpg}
    \caption{Hình ảnh thực tế vi điều khiển STM32F103C8T6}
    \label{fig:stm32}
\end{figure}
\subsubsection{Khóa chốt điện từ LY-03}
Solenoid Lock LY-03, hoạt động như một ổ khóa cửa sử dụng Solenoid để kích đóng mở bằng dòng điện. Trong đề tài này, nhóm sử dụng biến áp và relay 5V để cấp,
ổn áp dòng điện để truyền tín hiệu từ vi điều khiển tới khóa chốt.
\begin{figure}[H]
    \centering
    \begin{subfigure}{0.3\textwidth}
        \centering
        \includegraphics[scale = 0.8]{picture/solenoid.png}
        \label{fig:image1}
    \end{subfigure}
    \begin{subfigure}{0.6\textwidth}
        \centering
        \includegraphics[scale = 0.8]{picture/image.png}
        \label{fig:image2}
    \end{subfigure}
    \caption{Hình ảnh thực tế (trái) và nguyên lý hoạt động (phải) của Solenoid}
    \label{fig:both}
\end{figure}
\vspace{-20pt}
\begin{table}[H]
    \centering
    \begin{adjustbox}{width=0.7\textwidth}
    \begin{tabular}{|>{\centering\arraybackslash}m{6cm}|>{\centering\arraybackslash}m{3cm}|}    
        \hline
        \textbf{Thông số} & \textbf{Giá trị} \\ \hline
        Điện áp sử dụng & 12V DC \\ \hline
        Dòng điện tiêu thụ & 0.8A \\ \hline
        Công suất tiêu thụ & 9.6W \\ \hline
        Loại Solenoid & Solenoid từ \\ \hline
        Tốc độ phản ứng & dưới 1s \\ \hline
        Thời gian kích liên tục & dưới 10s \\ \hline
    \end{tabular}
    \end{adjustbox}
    \caption{Thông số khóa điện LY-03}
\end{table}
\subsubsection{Màn hình LED LCD 16 $\times$ 2}
Màn hình với nền xanh dương sử dụng driver HD44780, có khả năng hiển thị 2 dòng với mỗi dòng 16 ký tự. 
Kết hợp với mạch chuyển giao tiếp I\textsuperscript{2}C để dễ dàng truyền nhận dữ liệu với bộ xử lý trung tâm.
\begin{table}[H]
    \centering
    \begin{adjustbox}{width=0.7\textwidth}
    \begin{tabular}{|>{\centering\arraybackslash}m{6cm}|>{\centering\arraybackslash}m{6cm}|}    
        \hline
        \textbf{Thông số} & \textbf{Giá trị} \\ \hline
        Điện áp sử dụng & 5V DC \\ \hline
        Kích thước & 80 $\times$ 36 $\times$ 12.5 mm \\ \hline
        Hiển thị & Chữ trắng, nền xanh dương \\ \hline
    \end{tabular}
    \end{adjustbox}
    \caption{Thông số của màn hình LED 16 $\times$ 2}
\end{table}
\begin{figure}[H]
    \centering
    \begin{subfigure}{0.3\textwidth}
        \centering
        \includegraphics[scale = 0.2]{picture/lcd.png}
        \label{fig:lcd}
    \end{subfigure}
    \begin{subfigure}{0.6\textwidth}
        \centering
        \includegraphics[scale = 0.3]{picture/lcd_pin.png}
        \label{fig:lcd2}
    \end{subfigure}
    \caption{Hình ảnh thực tế (trái) và bản đồ chân cắm (phải) của màn hình LED}
    \label{fig:lcdlcd}
\end{figure}
\subsubsection{Bàn phím ma trận 4 $\times$ 4}
Bàn phím ma trận (Matrix Keypad) 4 $\times $4 là một giải pháp nhập liệu rời rạc phổ biến, 
cung cấp khả năng nhập dữ liệu số và lệnh điều khiển thông qua mười sáu phím bấm vật lý, được 
sắp xếp theo cấu trúc lưới không gian hai chiều. Ưu điểm cốt lõi của kiến trúc này là khả năng 
tối ưu hóa tài nguyên phần cứng (GPIO Optimization), cho phép quản lý mười sáu trạng thái logic độc 
lập với số đường tín hiệu chỉ bằng một nửa số trạng thái. Các phím bấm không có nguồn điện riêng 
và chia sẻ đường tín hiệu chung, vi điều khiển không thể đọc trực tiếp trạng thái của tất cả mười sáu
phím cùng một lúc. Thay vào đó, hệ thống phải sử dụng kỹ thuật quét lên toàn bộ bàn phím, với mỗi lần quét được thực thi cách nhau
một khoảng thời gian rất ngắn cố định để đảm bảo ghi nhận được mọi dữ liệu mà người dùng đưa vào từ bàn phím.
\begin{figure}[H]
     \centering
    \begin{subfigure}{0.3\textwidth}
        \centering
        \includegraphics[scale = 0.3]{picture/keypad.png}
        \label{fig:keypad}
    \end{subfigure}
    \begin{subfigure}{0.6\textwidth}
        \centering
        \includegraphics[scale = 0.3]{picture/keypad_internal.png}
        \label{fig:ikeypad}
    \end{subfigure}
    \caption{Hình ảnh thực tế và sơ đồ nút (từ trái qua phải) của bàn phím}
    \label{fig:key}
\end{figure}

\begin{table}[H]
    \centering
    \begin{adjustbox}{width=0.7\textwidth}
    \begin{tabular}{|>{\centering\arraybackslash}m{6cm}|>{\centering\arraybackslash}m{4cm}|}    
        \hline
        \textbf{Thông số} & \textbf{Giá trị} \\ \hline
        Độ dài dây cắm & 88mm \\ \hline
        Số đầu nối & 8 chân \\ \hline
        Kích thước bàn phím &  77 $\times $ 69 mm \\ \hline
        Nhiệt độ hoạt động & \SI{0}{\celsius} đến \SI{70}{\celsius} \\ \hline
    \end{tabular}
    \end{adjustbox}
    \caption{Thông số của bàn phím ma trận}
\end{table}

\subsubsection{Các thành phần khác}

\vspace{-10pt}
\begin{table}[H]
    \centering
    \begin{adjustbox}{width=0.8\textwidth}
    \begin{tabular}{|>{\centering\arraybackslash}m{7cm}|>{\centering\arraybackslash}m{7cm}|}    
        \hline
        \textbf{Tên linh kiện} & \textbf{Công dụng} \\ \hline
        Jumper đực-đực, đực-cái, cái-cái & Kết nối các linh kiện\\ \hline
        Relay 5V & Ổn định điện áp cho Solenoid, nhận tín hiệu từ vi điều khiển \\ \hline
        Mạch giao tiếp I\textsuperscript{2}C & Hiện thực giao tiếp giữa màn hình LCD và vi điều khiển \\ \hline
    \end{tabular}
    \end{adjustbox}
    \caption{Các thành phần khác của ổ khóa}
\end{table}

\subsection{Giải thuật Knutt-Morris-Pratt}

Thuật toán \code{Knuth-Morris-Pratt} (KMP) được phát triển bởi Donald Knuth, Vaughan Pratt, và James Morris vào năm 1977. 
Nó được ứng dụng rộng rãi trong các công cụ tìm kiếm, trình biên dịch, và trình soạn thảo văn bản. Đây là một thuật toán 
tìm kiếm chuỗi hiệu quả, được dùng để tìm một mẫu trong một đoạn văn bảnTiền xử lý – Xây dựng mảng LPS (Longest Prefix Suffix). Thuật toán này sử dụng bước tiền xử lý để xử lý các 
trường hợp không khớp, nhờ đó đạt được độ phức tạp thời gian tuyến tính O(n).

Thuật toán được triển khai với hai bước như sau:
\vspace{-10pt}
\begin{enumerate}[label = \textbf{Bước \arabic*:}, leftmargin=*]
    \item \textbf{Tiền xử lý - Xây dựng mảng LPS (Longest Prefix Suffix)} \\
        Ta xử lý chuỗi mẫu để tạo ra một mảng LPS. Mảng có ý nghĩa là: “Nếu ký tự không khớp tại vị trí này, 
        ta có thể lùi lại bao nhiêu ký tự trong mẫu mà vẫn không bỏ lỡ khả năng khớp nào?” Nhờ đó, ta không cần phải quay 
        lại đầu mẫu mỗi khi gặp lỗi không khớp. Bước này chỉ được thực hiện một lần duy nhất, trước khi bắt đầu tìm kiếm trong chuỗi văn bản.
    \item \textbf{So sánh - Tìm chuỗi con trong đoạn chuỗi} \\
        Ta sẽ bắt đầu so sánh các ký tự trong chuỗi nhập vào lần lượt một ký tự:
        \vspace{-5pt}
        \begin{itemize}
            \item Nếu ký tự khớp: di chuyển tiếp đến vị trí tiếp theo của cả chuỗi mẫu và chuỗi nhập vào.
            \item Nếu ký tự không khớp, sẽ có hai trường hợp:
            \begin{itemize}
                \item Nếu không bắt đầu ở ký tự đầu tiên ở ký tự mẫu, ta dùng giá trị mảng LPS ở vị trí trước để đưa con 
                    trỏ chuỗi mẫu về vị trí đó. Điều đó có nghĩa là: nhảy về đến tiền tố dài nhất đồng thời cũng là hậu tố - không phải kiểm tra lại các ký tự đó.
                \item Nếu đó là ký tự đầu, ta chỉ cần chuyển tới vị trí tiếp theo của chuỗi vào để kiểm tra.
            \end{itemize}
            \item Nếu tới cuối chuỗi mẫu, thì ta đã tìm thấy chuỗi đó. Ta sẽ ghi nhận lại vị trí bắt đầu và tiếp tục kiểm tra.
        \end{itemize}
\end{enumerate}