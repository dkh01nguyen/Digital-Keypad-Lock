\section{Thiết kế}
Sơ đồ bao gồm bốn khối: \textit{Khối xử lý trung tâm; Khối giao diện người dùng; Khối truyền động và cảm biến và Khối quản lý năng lượng}.

\begin{figure}[H]
    \centering
    \includegraphics[scale = 1.2]{picture/blockd.png}
    \caption{Sơ đồ khối tổng quát}
    \label{block-diagram}
\end{figure}

\subsection{Khối xử lý trung tâm}
Đây là đầu não hành vi của toàn bộ ổ khóa. Vi điều khiển thực hiện nhiệm vụ tiếp nhận tín hiệu đầu vào, 
xử lý thuật toán xác thực và phát lệnh điều khiển đến các thiết bị ngoại vi. 

Khi nhập mật khẩu, vi điều khiển nhận chuỗi ký tự từ bàn phím, sau đó thực hiện tìm kiếm mật khẩu thực liên tục 
trong chuỗi dữ liệu nhập vào bằng thuật toán xác thực và kích hoạt cơ cấu mở khóa nếu nội dung được kiểm tra là chính xác. Trong trường hợp người 
dùng sử dụng phương thức mở khóa bằng chìa cơ, vi điều khiển sẽ đọc trạng thái từ microswitch gắn với ổ khóa cơ học để thực hiện hành động tương ứng để mở khóa.

Sau khi nhập mật khẩu, đơn vị xử lý trung tâm sẽ phát lệnh điều khiển đến các thiết bị đầu ra: màn hình hiển thị, còi báo động
và relay kích hoạt solenoid ứng với từng trạng thái hoạt động của hệ thống như: tiến hành mở khóa khi đã nhập đúng;
sai mật khẩu nhiều lần; hiển thị trực quan và bảo mật nội dung đang nhập... Song song với việc xử lý mở khóa, MCU đảm nhận cả việc giám sát và 
quản lý các thông số vận hành cần thiết.

\subsection{Khối giao diện người dùng}
\subsubsection*{Thiết bị đầu vào}
Giao diện nhập liệu của ổ khóa gồm những thành phần sau:
\begin{itemize}
    \item Bàn phím ma trận kích thước 4 $\times$ 4 cho phép người dùng nhập mật khẩu và các lệnh điều khiển thông qua các 
        phím số từ "0" đến "9" và các ký tự chữ cái từ “A” đến “F”, cùng với hai nút bấm rời hiện thực nút "Delete" và "Enter".
    \item Một microswitch cơ học được lắp đặt tại vị trí ổ khóa cơ có chức năng phát hiện và thông báo cho MCU khi 
        người dùng thực hiện thao tác mở khóa bằng chìa khóa truyền thống. 
    \item Nút bấm được gắn ở mặt trong của cửa, khi được kích hoạt sẽ truyền tín hiệu mở khóa đến MCU. 
\end{itemize}
    
\subsubsection{Thiết bị đầu ra} 
Cơ cấu chấp hành bao gồm:
\begin{itemize}
    \item Màn hình LCD hiển thị các trạng thái của khóa.
    \item Còi báo động phát ra các tín hiệu âm thanh trong các tình huống như phát hiện nhập sai mật khẩu liên tiếp nhiều lần.
    \item Hệ thống đèn LED chỉ thị trạng thái được đặt ở cả hai mặt trước và sau của khóa, giúp người dùng 
        quan sát trực quan trạng thái khóa của cửa.
\end{itemize}
\subsection{Khối truyền động và cảm biến}
\begin{itemize}
    \item Khóa điện từ kiểu solenoid đóng vai trò là cơ cấu chấp hành chính, thực hiện động tác đóng và mở chốt 
        khóa theo chỉ thị từ vi điều khiển.
    \item Solenoid yêu cầu nguồn cấp có công suất lớn hơn khả năng cung cấp trực tiếp của MCU, 
        vi điều khiển sẽ điều khiển solenoid gián tiếp thông qua một relay trung gian, tín hiệu điều khiển mức 
        logic 5V từ chân GPIO của MCU kích hoạt cuộn dây relay.
    \item Về phần cảm biến, hệ thống sử dụng một microswitch cảm biến trạng thái cửa nhận biết vị trí đóng hoặc mở 
        của cánh cửa và gửi về MCU, phục vụ cho thao tác cảnh báo người dùng.
\end{itemize}
\subsection{Khối quản lý năng lượng:}
Trong thiết kế hệ thống, khối quản lý năng lượng được xây dựng dựa trên hai phương án nhằm đảm bảo tính linh hoạt giữa lý thuyết và thực thi thực tế:
\begin{itemize}
    \item \textbf{Mô hình thiết kế lý thuyết (Giả lập):} 
    \begin{itemize}
        \item Hệ thống được thiết kế với kịch bản sử dụng nguồn điện từ bộ bốn viên pin AA mắc nối tiếp, tạo ra điện áp khoảng 6V.
        \item Theo thiết kế này, nguồn pin sẽ đi qua các mạch điều chỉnh điện áp để tạo ra các mức điện áp ổn định 5V hoặc 3.3V cung cấp cho MCU cùng các linh kiện điện tử khác.
        \item Đồng thời, hệ thống tích hợp mạch giám sát điện áp hoạt động trên nguyên tắc đo điện áp bộ nguồn và chuyển đổi thành tín hiệu số gửi về MCU, giúp đưa ra các thông báo pin yếu đến người dùng khi điện áp thấp hơn định mức.
    \end{itemize}

    \item \textbf{Hiện thực thực tế:} 
    \begin{itemize}
        \item Để đảm bảo tính ổn định cao trong quá trình triển khai đồ án và đáp ứng nhu cầu năng lượng lớn của cơ cấu chấp hành, nhóm đã hiện thực nguồn cấp thông qua các bộ biến áp (adapter) rời thay vì sử dụng pin.
        \item \textbf{Biến áp 12V:} Cung cấp nguồn động lực trực tiếp cho khóa chốt Solenoid LY-03 thông qua relay. Nguồn này đảm bảo dòng điện tiêu thụ 0.8A và công suất 9.6W cần thiết để kích hoạt cơ chế chốt khóa.
        \item \textbf{Biến áp 5V:} Cung cấp nguồn nuôi cho khối xử lý trung tâm và các thiết bị hiển thị, đảm bảo hệ thống vận hành liên tục và ổn định trên nền tảng breadboard mà không phụ thuộc vào dung lượng pin trong quá trình kiểm thử.
    \end{itemize}
\end{itemize}

